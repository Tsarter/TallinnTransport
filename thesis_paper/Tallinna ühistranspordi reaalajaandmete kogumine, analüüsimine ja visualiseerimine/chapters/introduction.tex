Ühistranspordi sõidukid suudavad vedada kordades rohkem inimesi kui samapalju ruumi võtvad autod. Kui aastal 2013 kasutas peamise liikumisviisina ühistransporti 62\% tallinlastest, siis aastal 2023 oli see arv 42\%. Samal ajal on autoga liiklejate protsent kasvanud 29\% pealt 39\% peale ehk 35\% kasv \cite{tallinlasteRahalolukusitlus2020},  \cite{tallinlasteRahalolukusitlus2023}. Kuna linnas on ruum piiratud ja autoteid ei saa lõputult laiendada, siis on see suurendanud ummikuid. Ning isegi kui kuskil on radu laiendatud, siis on see tulnud millegi arvelt, tihti elukeskkonna. Osalt seetõttu on Tallinna linnavalitsus seadnud eesmärgiks suurendada ühistranspordi kasutatavust, et vähendada liiklusummikuid ja parandada linnaelanike elukvaliteeti \cite{tallinn2035}. Selle eesmärgi saavutamiseks peab aga ühistransport muutuma paremaks.

Täna on Tallinna linnal olemas hea arusaam autoliiklusest. Autoliikluse hindamiseks on loodud mitmeid analüütilisi tööriistu ja ülelinnalisi autoliikluse mudeleid. Kahjuks on ühistranspordi jaoks vastavaid tööriistu oluliselt vähem või puuduvad need üldse. See raskendab ka otsuste tegemist, mis toetaksid ühistranspordi arengut.

Üks selline puuduv tööriist on ka autori lahendatav ülesanne. Tallinna Tehnikaülikooli transpordiplaneerimise professori Dago Antoviga konsulteerides on selgunud, et Tallinnas puuduvad ühistranspordi kiirust ja hilinemisi visualiseerivad tööriistad. Näiteks ei ole teada, kus on liikumiskiirused aeglasemad. Otsused tehakse suuresti vaatluste põhjal. Kui linnal oleksid paremad teadmised, siis need võiks aidata näidata, kuhu oleks ilmtingimata vaja uusi bussiradu ja kuhu paremat järelvalvet bussiradadel toimuvate rikkumiste jaoks. Lisaks kuhu oleks vaja tarku foore, mis näiteks annaks prioriteedi hilinevale bussile. Kui linn saaks kasutada paremaid tööriistu ja andmeid, võiks see kaasa aidata ühistranspordi kiiremaks ja usaldusväärsemaks muutmisele. Samuti saaks avalikus juhtida tähelepanu probleemsetele kohtadele.
