Töö eesmärgiks oli visualiseerida muidu raskesti arusaadavaid ühistranspordi  asukohaandmeid ja muuta need paremini kättesaadavaks.

Tegemise käigus ilmnes, kui palju keerulisem on suuremahuliste andmete hoiustamine. See põhjustas mitmeid kitsaskohti, mille käigus tuli proovida mitmeid andmete hoiustamise viise. Vastasel korral oleks osade päringute tegemine paari sekundi asemel võtnud mitusada korda rohkem aega. Töö algusega võrreldes suudeti andmete kogumisel andmemahtu vähendada 90\% andmete sisu muutmata.

Töö algul seadsin paika kaks küsimust. Esiteks, et mis on liinil sõitvate sõidukite liikumiskiirused. 
Sellele saadi vastus läbi graafilise kaardi, kust on näha sõidukite liikumiskiirused. Lisaks on sinna lisatud võimalused filtreerida andmeid selliselt, et on näha vaid huvipakkuv liin, kellaaeg, ajaakna suurus ja on võimalus jätta välja kiirused bussipeatuste lähedal. Arvutati ka välja ühe kuu keskmine liikumiskiirus iga liini kohta eraldi.

Teiseks, et kui suured on liinil sõitvate sõidukite hilinemised.
Selle saavutamine täielikult ei õnnestunud. Sõidugraafikute tõlgendamisega esines probleeme ja alles töö lõpus leiti teoorias sobiv lahendus. Sellegipoolest näidati töös võimalusi hilinemiste nägemiseks. Seda siis, kui kasutada ühte loodud tööriista, kus saab valida geograafilised asukohad ja vaadelda ajakulu nende vahel.

Lõppkokkuvõttes valmis veebileht, mis on avalikult kättesaadav, võrdlemisi kiire ja interaktiivne. Seal on mitmeid võimalusi Tallinna ühistranspordi asukohaandmete vaatlemiseks ja analüüsimiseks. Kõik kogutud andmed on avalikud kättesaadavad ja igaüks saab sinna peale ennast huvitava veebilehe ehitada. Selliselt võib tööst kasu olla nii ühistranspordihuvilistel, linnaplaneerijatel kui ka avalikusel.




